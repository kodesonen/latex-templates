\section{Introduksjon til oppgavesett}

\textbf{Velkommen til kodekveld!} \\

Kodesonen har et formål å skape et trivelig, teknologisk og utfordrende miljø for våre medlemmer. Er du ikke medlem enda, kan du når som helst registrere deg på \url{kodesonen.no/medlem} helt gratis. Vi jobber kontinuerlig med å appellere til våre medlemmer og å være så inkluderende som mulig. Som medlem vil du blant annet kunne bli med i vår Discord server på \url{kodesonen.no/discord}, hvor vi tilbyr både kodehjelp og et digitalt samtalerom for våre medlemmer. På våre hjemmesider vil du også finne en kurskatalog, vår medlemsliste, informasjon om våre utfordring og om oss. \\

For denne kvelden har vi laget et oppgavesett som er delt opp i flere forskjellige oppgaver. Oppgavene er sortert i en stigende rekkefølge etter vanskelighetsgrad, men trenger ikke å være fullført i samme rekkefølge. Det vil si at du står fritt frem til å plukke og velge fra hvilke oppgaver du liker. Trenger du hjelp til å finne en oppgave som passer deg eller hjelp med å løse en oppgave, kan du ta kontakt med en av våre mentorer. Våre mentorer vil være synlig ved at de har på seg et Kodesonen plagg. \\

Når vi løser tekniske oppgaver sammen fokuserer vi mest på C++ og Python, slik at vi alle blir flinkere i disse språkene. IDEene (Integrated Development Environment) vi anbefaler for Python er Anaconda Spyder eller PyCharm, til C++ anbefaler vi å bruke Microsoft Visual Studio. Om du derimot ønsker å bruke et annet språk eller IDE står du fritt til å velge akkurat hva du ønsker. \\

Foruten om det håper vi at alle kan være med på å inkludere hverandre og etablere et hyggelig miljø sammen. Selv for at kodekvelden er basert på et teknisk grunnlag håper Kodesonen at det også skal være en anledning for en sosial begivenhet. På denne måten kan vi sammen trene på å samarbeide med andre mennesker og jobbe opp mot et bestemt mål. \\

Har du noen spørsmål går det an å ta kontakt med en av våre mentorer. Det er også en mulighet for å sende oss en melding på \url{kontakt@kodesonen} eller på en av våre sosiale medier når som helst. Vi er alltid åpne for prat, spørsmål og tilbakemeldinger. \\

Vi håper du får en fin kveld! \\
